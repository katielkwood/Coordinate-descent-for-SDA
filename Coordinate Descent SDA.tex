\documentclass[12pt, reqno]{amsart}

\usepackage{hyperref}
\usepackage{amssymb,upref,enumerate}
\usepackage{tikz}
\usepackage{amsmath,amssymb,yfonts,amscd,amsthm,amsxtra,mathrsfs}
\usepackage{latexsym}
\usepackage{multirow}
\usepackage{graphicx}
\usepackage{tikz}
\usepackage{pgfplots}
\usepackage{xcolor}
\usepackage{matlab-prettifier}
\usepackage{tabto}



\usetikzlibrary{shapes.geometric, arrows}
\usepgfplotslibrary{polar}
\usepgflibrary{shapes.geometric}
\usetikzlibrary{calc}

%\usepackage[dvips]{graphics,epsfig}
%\usepackage{showkeys}

\headheight=8pt
\topmargin=0pt
\textheight=624pt
\textwidth=432pt
\oddsidemargin=18pt
\evensidemargin=18pt


\def\Xint#1{\mathchoice 
{\XXint\displaystyle\textstyle{#1}}% 
{\XXint\textstyle\scriptstyle{#1}}% 
{\XXint\scriptstyle\scriptscriptstyle{#1}}% 
{\XXint\scriptscriptstyle\scriptscriptstyle{#1}}% 
\!\int} 
\def\XXint#1#2#3{{\setbox0=\hbox{$#1{#2#3}{\int}$} 
\vcenter{\hbox{$#2#3$}}\kern-.5\wd0}} 
\def\ddashint{\Xint=} 
\def\dashint{\Xint-}


\newcommand{\ra}{\rightarrow}
\newcommand{\lra}{\longrightarrow}
\newcommand{\bey}{\begin{eqnarray*}}
\newcommand{\eey}{\end{eqnarray*}}
\newcommand{\ba}{\begin{align}}
\newcommand{\ea}{\end{align}}
\newcommand{\bea}{\begin{align*}}
\newcommand{\ena}{\end{align*}}
\newcommand{\be}{\begin{equation}}
\newcommand{\ee}{\end{equation}}
\newcommand{\R}{\mathbb R}
\newcommand{\Z}{\mathbb Z}
\newcommand{\Rec}{\mathcal R}
\newcommand{\C}{\mathbb C}
\newcommand{\N}{\mathbb N}
\newcommand{\B}{\mathcal B}
\newcommand{\I}{\mathcal I}
\newcommand{\Ca}{\mathcal C}
\newcommand{\D}{\mathcal D}
\newcommand{\F}{\mathbb F}
\newcommand{\A}{\mathcal A}
\newcommand{\Si}{\mathcal S }
\newcommand{\Sp}{\mathbb S }
\newcommand{\M}{\mathcal M }
\newcommand{\Nc}{\mathcal N }
\newcommand{\Hi}{\mathcal H }
\newcommand{\K}{\mathcal K }
\newcommand{\T}{\mathbb T }
\newcommand{\V}{\mathcal V }
%\newcommand{\H}{\mathcal H }
\newcommand{\Q}{\mathbb Q }
\newcommand{\Li}{\mathcal L }
\newcommand{\G}{\mathcal G}
\newcommand{\Ra}{\Rightarrow}
\newcommand{\Lra}{\Longrightarrow}
\newcommand{\ep}{\epsilon}
\newcommand{\dss}{\displaystyle}
\newcommand{\bc}{\begin{center}}
\newcommand{\ec}{\end{center}}
\newcommand{\fa}{\forall}
\newcommand{\vp}{\varphi}
\newcommand{\mL}{L^p_tL^q_x}
\newcommand{\mLR}{L^p_tL^q_x(\R^{n+1})}
\newcommand{\intR}{\int_{\R}}
\newcommand{\intRn}{\int_{\R^n}}
\newcommand{\intRno}{\int_{\R^{n+1}}}
\newcommand{\intRnm}{\int_{(\R^n)^m}}
\newcommand{\p}{\partial}
\newcommand{\lr}{\ d\lambda(r)}
\newcommand{\gs}{\ d\gamma(s)}
\newcommand{\mx}{\ d\mu(x)}
\newcommand{\ny}{\ d\nu(y)}
\newcommand{\BMO}{\mathcal{BMO}}
\newcommand{\va}{\vec{\alpha}}
\newcommand{\vf}{\vec{f}}
\newcommand{\vy}{\vec{y}}
\newcommand{\al}{\alpha}
\newcommand{\lan}{\langle}
\newcommand{\ran}{\rangle}
\newcommand{\avgQ}{\frac{1}{|Q|}\int_Q}
\newcommand{\avgR}{\frac{1}{|R|}\int_R}
\newcommand{\avgB}{\frac{1}{|B|}\int_B}
\newcommand{\avgI}{\frac{1}{|I|}\int_I}
\newcommand{\dl}{\displaylimits}
\DeclareMathOperator{\re}{Re}
\DeclareMathOperator{\im}{Im}


\newtheorem{theorem}{Theorem}[section]
\newtheorem{lemma}[theorem]{Lemma}
\newtheorem{corollary}[theorem]{Corollary}
\newtheorem{prop}[theorem]{Proposition}



\theoremstyle{definition}
\newtheorem*{ack}{Acknowledgments}
\newtheorem{example}[theorem]{Example}
\newtheorem{exercise}[theorem]{Exercise}
\newtheorem{defn}[theorem]{Definition}
\theoremstyle{remark}
\newtheorem{remark}[theorem]{Remark}


%\numberwithin{equation}{section}


\tikzstyle{process} = [rectangle, minimum width=3cm, minimum height=1cm, text centered, draw=black, fill=orange!30]
\tikzstyle{startstop} = [rectangle, rounded corners, minimum width=3cm, minimum height=1cm,text centered, draw=black, fill=red!30]
\tikzstyle{io} = [trapezium, trapezium left angle=70, trapezium right angle=110, minimum width=3cm, minimum height=1cm, text centered, draw=black, fill=blue!30]
\tikzstyle{decision} = [diamond, minimum width=3cm, minimum height=1cm, text centered, draw=black, fill=green!30]
\tikzstyle{arrow} = [thick,->,>=stealth]
\tikzstyle{dot} = [circle ,fill=blue, scale=0.4]
\tikzstyle{knowndot} = [circle ,fill=pink, scale=0.4]
\tikzstyle{circlenode}=[circle,draw,font=\sffamily\Large\bfseries]
\tikzstyle{opennode}=[fill=white, inner sep=1pt]


\pgfplotsset{my style/.append style={axis x line=middle, axis y line=middle, xlabel={$x$}, ylabel={$y$}, axis equal }}


 \lstdefinestyle{text}
{basicstyle=\footnotesize}




\begin{document}

\title[]{Coordinate Descent for SDA}
\maketitle

\begin{equation*}
\underset{x}{\text{min}}\ h(x) := f(x) + \lambda \Omega (x) 
\end{equation*}

f is smooth,  $\Omega$ is a regularization function that may be nonsmooth.\\

Alg. 2 update step:
\begin{equation*}
\begin{split}
z_{i_k}^k &\leftarrow \underset{\chi}{\text{argmin}} (\chi - x_{i_k}^k)^T[\nabla f(x^k)]_{i_k} + \frac{1}{2 \alpha _k} \| \chi - x_{i_k}^k \|_2 ^2 + \lambda \Omega_i(x) \ \text{ for some } \alpha _k > 0 \\
x^{k+1} &\leftarrow x^k + (z_{i_k}^k - x_{i_k}^k) e_{i_k}
\end{split}
\end{equation*}

\hfill \\

For our problem: \\ 
$f(\beta) = \|Y \theta^{t+1} - X \beta \|^2 + \gamma \|\beta\|^2 \ $ with $\Omega = I \ $ where $ \| \| = \|  \|_2$\\
$\Omega(\beta) = \|\beta\|_1 \Rightarrow \Omega_i(\beta_i) = |\beta_i|$\\

Assume we already have $\beta^t$. We need to find $\beta^{t+1}$. Then the update step for our problem is 

\begin{equation*}
\begin{split}
z_{i_t}^t &\leftarrow \underset{\chi}{\text{argmin }} \chi [\nabla 				f(\beta^t)]_{i_t} - \beta_{i_t}^t [\nabla f(\beta^t)]_{i_t} + \frac{1}		{2 \alpha _t} \| \chi - \beta_{i_t}^t \|_2 ^2 + \lambda |\chi| \\
\beta^{t+1} &\leftarrow \beta^t + (z_{i_t}^t - \beta_{i_t}^t) e_{i_t} \\
\end{split}
\end{equation*}

\hfill \\
That is, we want to minimize 
\begin{equation*}
\begin{split}
& \chi [\nabla f(\beta^t)]_{i_t} - \beta_{i_t}^t [\nabla f(\beta^t)]_{i_t} + \frac{1}{2 \alpha _t} | \chi - \beta_{i_t}^t | | \chi - \beta_{i_t}^t | + \lambda | \chi |\\
\Rightarrow & \chi [\nabla f(\beta^t)]_{i_t} - \beta_{i_t}^t [\nabla f(\beta^t)]_{i_t} + \frac{1}{2 \alpha _t} (\chi ^2 - 2 \chi \beta_{i_t}^t + (\beta_{i_t}^t)^2) + \lambda | \chi |.\\
\end{split}
\end{equation*}

Note that $\chi [\nabla f(\beta^t)]_{i_t} - \beta_{i_t}^t [\nabla f(\beta^t)]_{i_t} + \frac{1}{2 \alpha _t} (\chi ^2 - 2 \chi \beta_{i_t}^t + (\beta_{i_t}^t)^2)$ is a quadratic with respect to $\chi$. So the minimum is found where the gradient with respect to $\chi$ is 0. But since we have the $\lambda |\chi|$ term, we must also take the subgradient of the absolute value which will give us an interval that includes 0.

\begin{equation*}
\Rightarrow 0 \in [\nabla f(\beta^t)]_{i_t} + \frac{1}{\alpha_t	} \chi - \frac{1}{\alpha_t} \beta_{i_t}^t + \partial \lambda | \chi |
\end{equation*}

Say $\chi ^*$ is the optimal solution. We have three cases.

\begin{equation*}
\begin{split}
\chi ^* > 0 &\Rightarrow [\nabla f(\beta^t)]_{i_t} + \frac{1}{\alpha_t	} \chi ^* - \frac{1}{\alpha_t} \beta_{i_t}^t + \lambda = 0 \\
			&\Rightarrow \chi ^* = \beta_{i_t}^t - \alpha_t [\nabla f(\beta^t)]_{i_t} - \alpha_t \lambda\\
\chi ^* < 0 &\Rightarrow [\nabla f(\beta^t)]_{i_t} + \frac{1}{\alpha_t	} \chi ^* - \frac{1}{\alpha_t} \beta_{i_t}^t - \lambda = 0 \\
			&\Rightarrow \chi ^* = \beta_{i_t}^t - \alpha_t [\nabla f(\beta^t)]_{i_t} + \alpha_t \lambda\\
\chi ^* = 0 &\Rightarrow 0 \in [\nabla f(\beta^t)]_{i_t} - \frac{1}{\alpha_t} (0) - \frac{1}{\alpha_t} \beta_{i_t}^t + \lambda [-1,1] \\
			&\Rightarrow - \lambda + [\nabla f(\beta^t)]_{i_t} - \frac{1}{\alpha_t} \beta_{i_t}^t \leq 0 \leq \lambda + [\nabla f(\beta^t)]_{i_t} - \frac{1}{\alpha_t} \beta_{i_t}^t \\
			&\Rightarrow -[\nabla f(\beta^t)]_{i_t} + \frac{1}{\alpha_t} \beta_{i_t}^t \in [-\lambda, \lambda] \\
			&\Rightarrow |\beta_{i_t}^t - \alpha_t [\nabla f(\beta^t)]_{i_t} |  \leq \alpha_t \lambda \\
\end{split}
\end{equation*}

Say $D = \beta_{i_t}^t - \alpha_t [\nabla f(\beta^t)]_{i_t}$. Then the optimal solution for our update step is 

\begin{equation*}
z_{i_t}^t = \left\{
			\begin{array}{ll}
				0 & \quad |D| \leq \alpha_t \lambda \\
				D - \alpha_t \lambda & \quad D > 0 \text{ and } |D| > \alpha_t \lambda \\
				D + \alpha_t \lambda & \quad D < 0 \text{ and } |D| > \alpha_t \lambda
			\end{array}
			\right.
\end{equation*}
which is the soft-thresholding operator $S(D, \alpha_t \lambda)$.

\hfill \\
Now to calculate $\beta^{t+1}$ we need $z_{i_t}^t - \beta_{i_t}^t$.\\

\begin{equation*}
z_{i_t}^t - \beta_{i_t}^t = \left\{
			\begin{array}{ll}
				0 - \beta_{i_t}^t  & \quad |D| \leq \alpha_t \lambda \\
				D - \alpha_t \lambda - \beta_{i_t}^t & \quad D > 0 
				\text{ and } |D| > \alpha_t \lambda \\
				D + \alpha_t \lambda - \beta_{i_t}^t & \quad D < 0 
				\text{ and } |D| > \alpha_t \lambda
			\end{array}
			\right.
\end{equation*}

\begin{equation*}
D - \alpha_t \lambda - \beta_{i_t}^t = \beta_{i_t}^t - \alpha_t [\nabla f(\beta^t)]_{i_t} - \alpha_t \lambda - \beta_{i_t}^t = -\alpha_t [\nabla f(\beta^t)]_{i_t} - \alpha_t \lambda
\end{equation*}

\begin{equation*}
D + \alpha_t \lambda - \beta_{i_t}^t = \beta_{i_t}^t - \alpha_t [\nabla f(\beta^t)]_{i_t} + \alpha_t \lambda - \beta_{i_t}^t = -\alpha_t [\nabla f(\beta^t)]_{i_t} + \alpha_t \lambda
\end{equation*}

\begin{equation*}
\Rightarrow z_{i_t}^t - \beta_{i_t}^t = \left\{
			\begin{array}{ll}
				- \beta_{i_t}^t  & \quad |D| \leq \alpha_t \lambda \\
				- \alpha_t [\nabla f(\beta^t)]_{i_t} - \alpha_t \lambda
				 & \quad D > 0 \text{ and } |D| > \alpha_t \lambda \\
				- \alpha_t [\nabla f(\beta^t)]_{i_t} + \alpha_t \lambda 
				& \quad D < 0 \text{ and } |D| > \alpha_t \lambda
			\end{array}
			\right.
\end{equation*}
\hfill \\
To calculate $[\nabla f(\beta^t)]_{i_t}$ we see that 

\begin{equation*}
\begin{split}
f(\beta^t) &= (Y \theta^{t+1} - X \beta^t)^T (Y \theta^{t+1} - X \beta^t) + \gamma(\beta^t)^T (\beta^t)\\
           &= (\theta^{t+1})^T Y^T Y \theta^{t+1} - (\theta^{t+1})^T Y^T X \beta^t - (\beta^t)^T X^T Y \theta^{t+1} + (\beta^t)^T X^T X \beta^t + \gamma(\beta^t)^T (\beta^t)\\
           &= (\theta^{t+1})^T Y^T Y \theta^{t+1} - 2(\beta^t)^T X^T Y \theta^{t+1} + (\beta^t)^T X^T X \beta^t + \gamma(\beta^t)^T (\beta^t)\\
[\nabla f(\beta^t)]_{i_t} &= [-2 X^T Y \theta^{t+1} + 2 X^T X \beta^t + 2 \gamma \beta^t]_{i_t}
\end{split}
\end{equation*}

\hfill \\
To find the $i_t$ component of $[\nabla f(\beta^t)]_{i_t}$ we have

$[2 \gamma \beta^t]_{i_t} = 2 \gamma \beta_{i_t}^t$\\

$[2 X^T X \beta^t]_{i_t} = 2 (X_{i_t})^T X \beta^t$ where $X_{i_t}$ denotes the $i_t$ column of $X$\\

$[-2 X^T Y \theta^{t+1}]_{i_t} = -2 (X_{i_t})^T Y \theta^{t+1}$\\

$\Rightarrow [\nabla f(\beta^t)]_{i_t} = -2 (X_{i_t})^T Y \theta^{t+1} + 2 (X_{i_t})^T X \beta^t + 2 \gamma \beta_{i_t}^t$\\

\hfill \\
Plugging this into $-\alpha_t [\nabla f(\beta^t)]_{i_t} \pm \alpha_t \lambda$ and $D$ we get

\begin{equation*}
\begin{split}
-\alpha_t [\nabla f(\beta^t)]_{i_t} \pm \alpha_t \lambda &= 
		2 \alpha_t (X_{i_t})^T Y \theta^{t+1} - 2 \alpha_t (X_{i_t})^T 
		X\beta^t - 2 \alpha_t\gamma\beta_{i_t}^t \pm \alpha_t \lambda\\
D = \beta_{i_t}^t - \alpha_t [\nabla f(\beta^t)]_{i_t} &= 
		\beta_{i_t}^t + 2 \alpha_t (X_{i_t})^T Y \theta^{t+1} 
		- 2 \alpha_t (X_{i_t})^T X \beta^t - 2 \alpha_t\gamma \beta_{i_t}^t			\\
\end{split}
\end{equation*}

Recall that after we calculate $z_{i_t}^t$ we calculate $\beta^{t+1}$ by
$\beta^{t+1} \leftarrow \beta^t + (z_{i_t}^t - \beta_{i_t}^t) e_{i_t}$. So only the $i_t$ component of $\beta^{t+1}$ is update. Therefore, the coordinate update for $\beta^{t+1}$ is 

\begin{equation*}
\begin{split}
\beta_{i_t}^{t+1} &= \beta_{i_t}^t + z_{i_t}^t - \beta_{i_t}^t \\
		&= \left\{
		\begin{array}{ll}
				\beta_{i_t}^t - \beta_{i_t}^t  & \quad |D| \leq 
				\alpha_t \lambda \\
				\beta_{i_t}^t + 2 \alpha_t (X_{i_t})^T Y \theta^{t+1} 
				- 2 \alpha_t (X_{i_t})^T X\beta^t - 2 \alpha_t 
				\gamma\beta_{i_t}^t - \alpha_t \lambda
				 & \quad D > 0 \text{ and } |D| > \alpha_t \lambda \\
				\beta_{i_t}^t + 2 \alpha_t (X_{i_t})^T Y \theta^{t+1} 
				- 2 \alpha_t (X_{i_t})^T X\beta^t - 2 \alpha_t 
				\gamma\beta_{i_t}^t + \alpha_t \lambda
				& \quad D < 0 \text{ and } |D| > \alpha_t \lambda
			\end{array}
			\right. \\
		&= \left\{
		\begin{array}{ll}
				0  & \quad |D| \leq \alpha_t \lambda \\
				(1-2 \alpha_t \gamma)\beta_{i_t}^t + 2 \alpha_t 
				(X_{i_t})^T Y \theta^{t+1} - 2 \alpha_t (X_{i_t})^T 
				X\beta^t - \alpha_t \lambda
				 & \quad D > 0 \text{ and } |D| > \alpha_t \lambda \\
				(1-2 \alpha_t \gamma)\beta_{i_t}^t + 2 \alpha_t 
				(X_{i_t})^T Y \theta^{t+1} - 2 \alpha_t (X_{i_t})^T 
				X\beta^t + \alpha_t \lambda
				& \quad D < 0 \text{ and } |D| > \alpha_t \lambda
			\end{array}
			\right. \\	
		&= S(D, \alpha_t \lambda)		
\end{split}
\end{equation*}
where $D = (1-2 \alpha_t \gamma)\beta_{i_t}^t + 2 \alpha_t (X_{i_t})^T Y \theta^{t+1} - 2 \alpha_t (X_{i_t})^T X\beta^t$

\newpage

\section{Coordinate Descent based on GLM Approach}
$f(\beta) = \|Y \theta^{t+1} - X \beta \|^2 + \gamma \|\beta\|^2 + \lambda \|\beta\|_1 \ $ with $\Omega = I \ $ where $ \| \| = \|  \|_2$\\

Let $y = Y \theta^{t+1}$. Then we have\\
\begin{equation*}
\begin{split}
f(\beta) &= \|y - X \beta \|^2 + \gamma \|\beta\|^2 + \lambda \|\beta\|_1 \text{ with } \Omega = I  \text{ where }  \| \| = \|  \|_2 \\
	&= \sum_{i = 1}^{n} (y_i - \sum_{k = 1}^{p} X_{ik} \beta_k )^2 + \gamma \sum_{k = 1}^{p} \beta_k^2 + \lambda \sum_{k = 1}^{p} |\beta_k|
\end{split}
\end{equation*}

Rewrite $f(\beta)$ with a partial residual.

\begin{equation*}
\begin{split}
f(\beta) &= \sum_{i = 1}^{n} (y_i - \sum_{k \neq j}^{} X_{ik} \beta_k - X_{ij} \beta_j )^2 + \gamma \sum_{k = 1}^{p} \beta_k^2 + \lambda \sum_{k = 1}^{p} |\beta_k|
\end{split}
\end{equation*}

Minimize w.r.t. $\beta_j$ (i.e. take the partial derivative w.r.t. $\beta_j$ and set equal to $0$).

\begin{equation*}
\begin{split}
0 &\in 2 \sum_{i = 1}^{n} (y_i - \sum_{k \neq j}^{} X_{ik} \beta_k - X_{ij} \beta_j )(-X_{ij}) + 2 \gamma \beta_j + \lambda \phi \\
 & \text{where $\phi$ is the subgradient of $|\beta_j|$.}\\
\end{split}
\end{equation*}


Solve for $\beta_k$.\\
\begin{equation*}
\begin{split}
2 \sum_{i = 1}^{n} (y_i - \sum_{k \neq j}^{} X_{ik} \beta_k - X_{ij} \beta_j )(X_{ij})  - \lambda \phi &= 2 \gamma \beta_j\\
2 \sum_{i = 1}^{n} (X_{ij}y_i - X_{ij}\sum_{k \neq j}^{} X_{ik} \beta_k - X_{ij}^2 \beta_j )  - \lambda \phi &= 2 \gamma \beta_j\\
2 (X_j)^T y - 2 X_j^T (X \beta - \beta_j X_j) - 2\sum_{i = 1}^{n} X_{ij}^2 \beta_j  - \lambda \phi &= 2 \gamma \beta_j\\
\text{ where $X_j$ is the $j^{th}$ column of X} & \\
2 (X_j)^T y - 2 X_j^T (X \beta - \beta_j X_j) - 2(n-1) \beta_j  - \lambda \phi &= 2 \gamma \beta_j\\
\text{ since $\sum_{i = 1}^{n} X_{ij}^2 = n - 1$ for our data} & \\
2 (X_j)^T y - 2 X_j^T (X \beta - \beta_j X_j)  - \lambda \phi &= 2 \gamma \beta_j  + 2(n-1) \beta_j\\
2 (X_j)^T y - 2 X_j^T (X \beta - \beta_j X_j)  - \lambda \phi &= 2 (\gamma + (n-1)) \beta_j\\
\end{split}
\end{equation*}
\hfill \\
Say $\beta_j^*$ is the optimal solution. Let $Z = 2 (X_j)^T y - 2 X_j^T (X \beta - \beta_j X_j)$. We have 3 cases.

\begin{equation*}
\begin{split}
\beta_j^* > 0 \Rightarrow \phi = 1 &\Rightarrow Z  - \lambda = 2 (\gamma + n-1) \beta_j^*\\
& \text{ if $Z > 0$ and $\lambda < |Z|$} \\
\beta_j^* < 0 \Rightarrow \phi = -1 &\Rightarrow Z  + \lambda = 2 (\gamma + n-1) \beta_j^*\\
& \text{ if $Z < 0$ and $\lambda < |Z|$} \\
\beta_j^* = 0 \Rightarrow \phi = [-1,1] &\Rightarrow Z  - \lambda[-1,1] \in 0\\
&\Rightarrow Z - \lambda \leq 0 \leq Z + \lambda\\
& \Rightarrow \lambda > |Z| \\
\end{split}
\end{equation*}

Therefore, $2(\gamma + n - 1)\beta_j^* = S(Z, \lambda)$ where $S(Z, \lambda)$ is the soft-thresholding operator with value\\

\begin{equation*}
\begin{split}
S(Z, \lambda)
		&= \left\{
		\begin{array}{ll}
				0  & \quad \lambda \geq |Z|\\
				Z - \lambda & \quad Z > 0 \text{ and } \lambda < |Z|\\
				Z + \lambda & \quad Z < 0 \text{ and } \lambda < |Z|
			\end{array}
			\right. \\				
\end{split}
\end{equation*}

Hence, 
\begin{equation*}
\beta_j^* = \frac{S(Z, \lambda)}{2(\gamma + n - 1)}
\end{equation*}

\end{document}

$$
f(x) = \left\{
        \begin{array}{ll}
            -x & \quad x \leq 0 \\
            x & \quad x > 0
        \end{array}
    \right.
$$